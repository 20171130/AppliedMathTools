% !TeX encoding = UTF-8
% !TeX program = LuaLaTeX
% !TeX spellcheck = en_US

% Author : pppppass
% Description : Materials: LaTeX --- Seminar on Selected Tools Week 0 --- Python, LaTeX and Git

\documentclass[english]{../TeXTemplate/pkupaper}

\usepackage[paper]{../TeXTemplate/def}

\newcommand{\cuniversity}{}
\newcommand{\cthesisname}{Materials: \texorpdfstring{\LaTeX}{LaTeX} --- Seminar on Selected Tools Week 0 --- Python, \texorpdfstring{\LaTeX}{LaTeX} and Git}
\newcommand{\titlemark}{Materials: \texorpdfstring{\LaTeX}{LaTeX}}

\title{\titlemark}
\author{pppppass}
\date{Updated on January 23, 2018}

\begin{document}

\maketitle

The information is updated on January 23, 2018.
 
\section{Installation and configuration}

For \TeX{} distribution, \href{http://www.tug.org/texlive/}{\TeX{} Live} is an official comprehensive TeX distribution system, which provides system-specific supports.

For Windows and Linux, reading through \href{http://www.tug.org/texlive/acquire-netinstall.html}{installing \TeX{} Live over the Internet} is recommended. A (non-necessary) introduction to \TeX{} Live on Windows is provided \href{http://www.tug.org/texlive/windows.html}{here}. If you are going to install \TeX{} Live on Linux, please read \href{http://www.tug.org/texlive/quickinstall.html}{this page} first, which gives detailed guidance. Installing on your home directory (\verb"/home/someone/texlive/2017" to be exact) instead of default \verb"/usr/local/texlive/2017" is recommended in case of authority issues.

For macOS, please install \href{http://www.tug.org/mactex/}{MacTeX}, which is specially adapted to macOS and includes \TeX{} Live.

Installation information can also be found in \emph{\LaTeX 入门}.

Note that there are also alternatives for installation, like \href{http://www.ctex.org/CTeX}{CTeX}, which a suite specialized to Chinese and can be downloaded \href{http://www.ctex.org/CTeXDownload}{here}. However, CTeX is somehow out-of-date now and have no advantages due to the development of XeTeX and LuaTeX.

Some Linux systems provides packages of \TeX{} Live in their software repository, but installing in this way may leads to a loss in integrity. That is, some packages and the documents provided by \TeX{} Live may be missing.

\section{Resources}

The book \href{https://item.jd.com/11258469.html}{\emph{\LaTeX 入门}} is a useful book for anyone using \LaTeX. Not only is the book a complete tutorial, it also works as an index to frequently used packages. Only chapter 1 is suggested for the first reading, while other chapters, which are filled with details, may be used as a manual. The outline slide is based on this book.

The Wikibook \href{https://en.wikibooks.org/wiki/LaTeX}{\emph{\LaTeX}} is featured on Wikibooks, providing a brief introduction to \LaTeX. There are partial Chinese translation.

Note that \TeX{} Live itself includes a documentation system, which can be accessed by the command \verb"texdoc <name>".

The famous \href{https://www.ctan.org/tex-archive/info/lshort/english/}{\emph{The not so Short Introduction to \LaTeX}} is an introduction of moderate length. There is also a \href{ https://www.ctan.org/tex-archive/info/lshort/chinese}{Chinese translation}. It can be directly accessed by \verb"texdoc lshort" and \verb"texdoc lshort-chinese".

By utilize \verb"texdoc", one may access document of packages and document classes. For example, executing \verb"texdoc ctex" on a terminal, the documentation of the package \verb"ctex" shows up.

Some other important reference can be accessed by \verb"texdoc". \href{https://www.ctan.org/tex-archive/info/symbols/comprehensive/}{\emph{The Comprehensive \LaTeX{} Symbol List}} can be accessed by \verb"texdoc comprehensive", which lists many symbols. \href{https://www.ctan.org/pkg/maths-symbols}{\emph{Summary of Mathmatical Symbols available in \LaTeX}} can be accessed by \verb"texdoc symbols", which is a compact summary of \LaTeX{} symbols. Further information of \TeX{} Live can be found by \verb"texdoc texlive".

For websites, \href{https://tex.stackexchange.com/}{TeX StackExchange} is a community for \TeX{} and \LaTeX{} users, which is very helpful for hard \TeX{} and \LaTeX{} questions and practical tricks.

\section{Assignment}

The assignment is listed below.

\begin{partlist}
\item \textbf{(Optional)} Article exercise: finish the example described in Section 1.2 in \emph{\LaTeX 入门}.
\item \textbf{(Required)} Article template: compose your own template for papers. Requirement is given in Question \ref{Ques:ArtTemp}.
\item \textbf{(Optional)} Beamer exercise: finish the example described in Section 6.1 in \emph{\LaTeX 入门}.
\item \textbf{(Optional)} Beamer template: compose your own beamer template for slides. Requirement is given in Question \ref{Ques:BeaTemp}
\item \textbf{(Optional)} Package: pack your paper template and slide template into a package. Requirement is given in Question \ref{Ques:Pack}.
\end{partlist}

\begin{thmquestion} \label{Ques:ArtTemp}
Compose your own template for papers. Such a template should include these components:
\begin{partlist}
\item Chinese support: use document class \verb"ctexart".
\item Layout configuration: use package \verb"geometry".
\item Theorem-like environments: use package \verb"ntheorem" with configuration.
\item List-like environments: use package \verb"enumitem" with configuration.
\item Program list environments: use package \verb"listings" with configuration.
\item Formula support: use package \verb"amsmath".
\end{partlist}
Make use of such template by \verb"\include{...}".
\end{thmquestion}

\begin{thmquestion} \label{Ques:BeaTemp}
Compose your own beamer template for slides. Such a template should include these components:
\begin{partlist}
\item Inner themes.
\item Outer themes.
\item Color themes. It is recommend to configure your own colors.
\item Font themes. Use sans serif fonts for normal texts and font theme \verb"professionalfonts" for mathematical formulas.
\end{partlist}
\end{thmquestion}

\begin{thmquestion} \label{Ques:Pack}
Pack your paper template and slide template into a package. Such package should be composed of \verb".cls" (document class) files and and \verb".sty" (style) files. Utilize this package by using \verb"\documentclass{...}" and \verb"\usepackage{...}" directly.
\end{thmquestion}

\end{document}
