\subsection{random}
This module implements pseudo-random number generators for various distributions. Almost all module functions depend on the basic function \textit{random}().

Several commonly used functions in \texttt{random}:
\begin{itemize}
\item \textbf{random()} returns a random float number in [0.0, 1.0).
\item \textbf{randrange(n), randrange(m,n), randrange(m,n,d)} returns a random integer in the interval (this refers to the definition of \textbf{range}).
\item \textbf{randint}$(m,n)$ is the same as \textbf{randrange}$(m,n+1)$.
\item \textbf{choice}$(s)$ selects a random character in the string $s$.
\item \textbf{seed(n), seed()} uses $n$ or system time to reset the random number generator. It is used to restart a random series.
\item \textbf{shuffle}$(x)$ shuffles the sequence $x$ in place.
\item \textbf{sample}$(population,k)$ returns a $k$ length list of unique elements chosen from the population sequence or set.
\end{itemize}
\texttt{random} also provides functions that generate specific real-valued distributions. You can get more detailed listing in \href{https://docs.python.org/3.6/library/random.html#real-valued-distributions}{9.6.4. Real-valued distributions}