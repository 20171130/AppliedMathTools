% !TeX encoding = UTF-8 Unicode
% !TeX program = LuaLaTeX
% !TeX spellcheck = LaTeX

% Author : pppppass
% Description : Matetials: Python Basics -- Seminar on Selected Tools Week 0

\documentclass[english]{pkupaper}

\usepackage[paper]{def}

\usepackage{lipsum}

\newcommand{\cuniversity}{}
\newcommand{\cthesisname}{Matetials: Python Basics -- Seminar on Selected Tools Week 0}
\newcommand{\titlemark}{Matetials: Python Basics}

\title{\titlemark}
\author{pppppass}
\date{\today}

	\begin{document}

	\maketitle

Everything is optional except that marked with \textbf{(Required)}. The information is updated on December 2, 2017.

\section{Installation and configuration}

There are Python distributions of Windows version, but I do not recommend because it takes time to get around the differences between Windows and Unix-like systems, including line feed (\verb"\n" in Linux but \verb"\n\r" in Windows), path separator (\verb"/" in Linux and \verb"\" in Windows) and so on. Additionally, some packages like NumPy and PyTorch does not work well in Windows.

Python is included in many Linux distributions, but I personally recommmend  use Anaconda to create virtual environments.

\section{Resources}

\subsection{Tutorials}

If you are familiar with C++ or some other object-oriented language, \href{https://docs.python.org/3.6/tutorial/index.html}{official Python Tutorial} should be sufficient. Details in Section 6.4, Section 9.5.1, Section 10--13, Section 15--16 can be skipped. The outline is based on this tutorial.

\href{https://www.liaoxuefeng.com/wiki/0014316089557264a6b348958f449949df42a6d3a2e542c000}{Xuefeng Liao's Python Tutorial} is also recommended if you cannot stand so much English. This tutorial covers more topics than the official Python Tutorial, but I consider some topics too detailed as a introductory tutorial instead of a manual.
The sections before IO Programming inclusive should be adequate.

Google provides a \href{https://developers.google.com/edu/python/}{Python Class} but this class does not cover any object-oriented or functional topics.

For books, I have heard of \href{https://item.jd.com/11681561.html}{\emph{Python Cookbook}} and \href{http://item.jd.com/11572056.html}{\emph{Learn Python the Hard Way}}.

Discussions on Zhihu provides lists of resources like \href{https://www.zhihu.com/question/29138020}{如何系统地自学 Python 的?} and \href{https://www.zhihu.com/question/20702054}{你是如何自学 Python 的?} \href{https://wiki.python.org/moin/BeginnersGuide/Programmers}{A list of resources} are also provided officially.

\subsection{PEPs}

It is strongly recommened to read through \href{https://www.python.org/dev/peps/pep-0020/}{\emph{PEP 20 -- The Zen of Python}} to get familiar with the ideas of Python.

I personally recommend to get through \href{https://www.python.org/dev/peps/pep-0008/}{\emph{PEP 8 -- Style Guide for Python Code}}. As component of a project, this guide are detailed, so pay attention not to be trapped in details. Note that PyCharm and some other IDEs have PEP8 Code formatting support.

\href{https://www.python.org/dev/peps/}{\emph{PEP 0 -- Index of Python Enhancement Proposals}} offers an insight to the whole Python project and community, and \href{https://www.python.org/dev/peps/pep-0007/}{\emph{PEP 7 -- Style Guide for C Code}} shows guidelines for the Python project.

\subsection{Websites}

\href{https://stackoverflow.com/}{Stack Overflow} is a community for programmers, and many Q\&A can be found. It's is suggested to search Stack Overflow first if some bugs occur.

Python has its own \href{https://python-forum.io/}{Forum} and \href{https://wiki.python.org/moin/FrontPage}{Wiki}.

\section{Assignment}

\begin{thmquestion}
\textbf{(Required)} Use a generator to generate all permutations of $n$ (given) numbers. Solution can be found in \verb"permutation.py".
\end{thmquestion}

\begin{thmquestion}
Use a generator to solve the eight queens puzzle. Further information can be found in \href{http://cssyb.openjudge.cn/2017finalpython/B/}{OpenJudge}. (The names in the code is a little confusing) Solution wanted.
\end{thmquestion}

\begin{thmquestion} \label{Ques:LS}
\textbf{(Required)} Let $ \rbr{ x_i, y_i } \in \Rset^2 $ where $ i = 1, 2, \cdots, n $. The least square regression problem is to find $a$ and $b$ such that
\begin{equation}
F \rbr{ a, b } = \frac{1}{2} \sume{i}{1}{n}{\rbr{ y_i - a - b x_i }^2}
\end{equation}
reaches its minimum. We're going to solve this problem by gradient method. That is, we fix some $a^{\rbr{0}}$ and $b^{\rbr{0}}$ first, and then update them by
\begin{gather}
a^{\rbr{ i + 1 }} = a^{\rbr{i}} - \eta \frac{ \pd F }{ \pd a }, \\
b^{\rbr{ i + 1 }} = b^{\rbr{i}} - \eta \frac{ \pd F }{ \pd b }.
\end{gather}
The file \verb"utils.py" in folder \verb"trainer" provides three functions. Function \verb"generate_config" generate values, \verb"loss_func" returns the value of $F$, and \verb"train_func" performs an update.
\begin{partlist}
\item Install NumPy first. This can be done by using Anaconda (recommended) or \verb"pip".
\item Encapsule these functions into a class \verb"Trainer", which provides interfaces to:
\begin{partpartlist}
\item set configurations. (This can also be done in initialization)
\item train for an arbitrary number of iterations.
\item print the result.
\end{partpartlist}
\end{partlist}
Solutions are provided in \verb"trainer.py" and \verb"trainer.ipynb".
\end{thmquestion}

\begin{thmquestion}
Try to add an interface to modify the step size $\eta$ in Question \ref{Ques:LS}. Can you implement dimishing step size ($ \eta^{\rbr{i}} = C_1 / \rbr{ i + C_2 } $)? Solution wanted.
\end{thmquestion}

Additional exercise can be found in the Practise part of \href{http://www.pyschools.com/}{PySchools}. Note that there are a huge number of exercises, and it is not recommended to finished them all.

\href{https://projecteuler.net/}{Project Euler} is a website offering mathematical problems. These problems may be solved by Python.

	\end{document}
