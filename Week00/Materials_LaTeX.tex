% !TeX encoding = UTF-8 Unicode
% !TeX program = LuaLaTeX
% !TeX spellcheck = LaTeX

% Author : pppppass
% Description : Matetials: LaTeX -- Seminar on Selected Tools Week 0

\documentclass[english]{pkupaper}

\usepackage[paper]{def}

\usepackage{lipsum}

\newcommand{\cuniversity}{}
\newcommand{\cthesisname}{Matetials: \LaTeX -- Seminar on Selected Tools Week 0}
\newcommand{\titlemark}{Matetials: \LaTeX}

\title{\titlemark}
\author{pppppass}
\date{\today}

	\begin{document}

	\maketitle

Everything is optional except that marked with \textbf{(Required)}. The information is updated on December 3, 2017.

\section{Installation and configuration}

\href{http://www.tug.org/texlive/}{TeX Live} is a official comprehensive TeX distribution system, which provides system-specific supports. I personally recommend to install TeX Live onto the most frequently used system.

For Windows and Linux, reading through \href{http://www.tug.org/texlive/acquire-netinstall.html}{installing TeX Live over the Internet} is recommended. An (non-necessary) introduction to TeX Live on Windows is provided \href{http://www.tug.org/texlive/windows.html}{here}. If you are going to install TeX Live on Linux, please read \href{http://www.tug.org/texlive/quickinstall.html}{this page} first. I recommend to install on your home directory \verb"~" (\verb"/home/someone/texlive/2017") instead of default \verb"/usr/local/texlive/2017" to avoid authority issues.

For Mac OS, please install \href{http://www.tug.org/mactex/}{MacTeX}, which is specially adapted to Mac OS and includes TeX Live.

Installation information can also be found in the book \emph{\LaTeX 入门}.

There are also alternatives for installation.

\href{http://www.ctex.org/CTeX}{CTeX} is a suite specialized to Chinese, which can be downloaded \href{http://www.ctex.org/CTeXDownload}{here}. However, CTeX is out-of-date now and have no advantages due to the development of XeTeX and LuaTeX.

Some Linux systems provides packages of TeX Live in their software repository, but installing in this way may leads to a loss in integrity. That is, some packages and the documents provided by TeX Live may be missing.

\section{Resources}

TeX Live includes a documentation system, which can be accessed by the command \verb"texdoc <name>".

\subsection{Tutorials}

I highly recommend the book \href{https://item.jd.com/11258469.html}{\emph{\LaTeX 入门}} for anyone using LaTeX. Not only is the book a complete tutorial, it also works as an index to frequently used packages. Only chapter 1 is suggested for the first reading, and other chapters, which are filled with details, may be used as a manual. The outline is based on this book.

The Wikibook \href{https://en.wikibooks.org/wiki/LaTeX}{\emph{\LaTeX}} is featured on Wikibooks, providing a brief introduction to \LaTeX. There are partial Chinese translation.

The famous \href{https://www.ctan.org/tex-archive/info/lshort/english/}{\emph{The not so Short Introduction to \LaTeX}} is an introduction of moderate length. There are also \href{ https://www.ctan.org/tex-archive/info/lshort/chinese}{Chinese translations}. It can be directly accessed by \verb"texdoc lshort" and \verb"texdoc lshort-chinese".

\subsection{References}

By utilize \verb"texdoc", one may access document of any packages and document classes. For example, using \verb"texdoc ctex", the documentation of the package \verb"ctex" shows up.

Some other important reference can be accessed by \verb"texdoc".

\href{https://www.ctan.org/tex-archive/info/symbols/comprehensive/}{\emph{The Comprehensive \LaTeX Symbol List}} can be accessed by \verb"texdoc comprehensive", which lists many symbols.

\href{https://www.ctan.org/pkg/maths-symbols}{\emph{Summary of Mathmatical Symbols available in \LaTeX}} can be accessed by \verb"texdoc symbols", which is a compact summary of \LaTeX symbols.

\href{https://www.ctan.org/tex-archive/info/symbols/comprehensive/}{\emph{The Comprehensive \LaTeX Symbol List}} can be accessed by \verb"texdoc comprehensive", which lists many symbols.

\href{https://www.ctan.org/pkg/free-math-font-survey}{\emph{A survey of available free Mathematics fonts}} is a list of free mathematics fonts.

Further information of TeX Live can be founded by \verb"texdoc texlive".

\subsection{Websites}

\href{https://tex.stackexchange.com/}{TeX StackExchange} is a community for \TeX and \LaTeX users.

\section{Assignment}

\begin{thmquestion}
Finish the example described in Section 1.2 in \emph{\LaTeX 入门}.
\end{thmquestion}

\begin{thmquestion}
\textbf{(Required)} Compose your own \LaTeX template for papers. Such a template should includes at least these components:
\begin{partlist}
\item Chinese support: use document class \verb"ctexart".
\item Layout configuration: use package \verb"geometry".
\item Theorem-like environments: use package \verb"ntheorem" and configure it.
\item List-like environments: use package \verb"enumitem" and configure it.
\item Program list environments: use package \verb"listings" and configure it.
\item Formula support: use package \verb"amsmath".
\end{partlist}
Make use of such template by \verb"\include{...}".
\end{thmquestion}

\begin{thmquestion}
Finish the example described in Section 6.1 in \emph{\LaTeX 入门}.
\end{thmquestion}

\begin{thmquestion}
Compose your own beamer template for slides. Such a template should includes at least these components:
\begin{partlist}
\item Innter themes.
\item Outer themes.
\item Color themes. It is recommend to configure your own colors.
\item Font themes. Use sans serif fonts for normal texts and font theme \verb"professionalfonts" for formulas.
\end{partlist}
\end{thmquestion}

\begin{thmquestion}
Typeset a paper and a slide using \LaTeX.
\end{thmquestion}

\begin{thmquestion}
Pack your paper template and slide template into \verb".cls" (document class) files and and \verb".sty" (style) files. Utilize them by using \verb"\documentclass{...}" and \verb"\usepackage{...}" directly.
\end{thmquestion}

	\end{document}
