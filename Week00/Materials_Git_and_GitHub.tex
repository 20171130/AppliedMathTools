% !TeX encoding = UTF-8 Unicode
% !TeX program = LuaLaTeX
% !TeX spellcheck = LaTeX

% Author : pppppass
% Description : Matetials: Git and GitHub -- Seminar on Selected Tools Week 0

\documentclass[english]{pkupaper}

\usepackage[paper]{def}

\usepackage{lipsum}

\newcommand{\cuniversity}{}
\newcommand{\cthesisname}{Matetials: Git and GitHub -- Seminar on Selected Tools Week 0}
\newcommand{\titlemark}{Matetials: Git and GitHub}

\title{\titlemark}
\author{pppppass}
\date{\today}

	\begin{document}

	\maketitle

Everything is optional except that marked with \textbf{(Required)}. The information is updated on December 3, 2017.

\section{Git}

\subsection{Installation and configuration}

For windows, there're \href{https://git-for-windows.github.io/}{Git for Windows} and it's fully functional. However, I personally recommend UNIX-like terminal emulation in Windows, like \href{http://www.msys2.org/}{MSYS2} and \href{https://www.cygwin.com/}{Cygwin}.

For Linux, git can always be retrived from the distribution repository. For example, \verb"apt" can be used for Ubuntu.

For Mac, \href{https://git-scm.com/book/en/v2/Getting-Started-Installing-Git}{this page} offers a detailed guide.

\subsection{Resources}

\subsubsection{Tutorials} \label{Sssec:Tuto}

\href{https://www.liaoxuefeng.com/wiki/0013739516305929606dd18361248578c67b8067c8c017b000/}{Xuefeng Liao's Git Tutorial} is recommended for beginners in Git. The outline is based on this tutorial.

Git officially offers the book \href{https://git-scm.com/book/en/v2}{\emph{Pro Git}}, which is also a great tutorial for Git. Note that \verb".pdf" format and Chinese version are also provided. This tutorial covers more topics than Xuefeng Liao's tutorial, and therefore first 5 sections are sufficient.

GitHub itself provides a training kit for Git (and GitHub as well), for which one may refer to \href{https://services.github.com/on-demand/}{On Demand Train} and \href{https://services.github.com/on-demand/resources/learning-path/}{Learning Path}. The corresponding Git repository is \href{https://github.com/github/training-kit}{here}.

Further information can be found in \href{https://git-scm.com/doc}{Git's official document}, where videos and a \href{https://services.github.com/on-demand/downloads/github-git-cheat-sheet.pdf}{cheat sheet} of Git commands are provided.

\subsubsection{Websites}

Questions about Git may be found in \href{https://stackoverflow.com/}{Stack Overflow}.

\subsection{Assignment}

\begin{thmquestion}
\textbf{(Required)} Initialize a Git repository for assignments locally. Add your assignments to the repository and commit your changes. Keep using Git to arrange assignments.
\end{thmquestion}

\section{GitHub}

\subsection{Installation and configuration}

GitHub is fully accessible through \href{https://github.com/}{GitHub}.

\href{https://desktop.github.com/}{GitHub for Windows} also exists, but it can do no more than a terminal (and Git).

\subsection{Resources}

\subsection{Tutorials}

GitHub itself provides a \href{https://guides.github.com/activities/hello-world/}{brief introduction} to GitHub which I highly recommend. \href{https://guides.github.com/}{GitHub Guides} covers more topics which you may find helpful at some point.

\href{https://www.liaoxuefeng.com/wiki/0013739516305929606dd18361248578c67b8067c8c017b000/00137628548491051ccfaef0ccb470894c858999603fedf000}{A section} in Xuefeng Liao's Git Tutorial introduces GitHub.

\href{https://git-scm.com/book/en/v2/GitHub-Account-Setup-and-Configuration}{Section 6} of \emph{Pro Git} also introduces GitHub.

GitHub itself provides a training kit for GitHub, which is introduced in Subsubsection \href{Sssec:Tuto}.

\subsubsection{Websites}

You may search \href{https://stackoverflow.com/}{Stack Overflow} for GitHub question.

Directly raising an issue on the seminar repository is always welcomed.

\subsubsection{Assignment}

\begin{thmquestion}
\textbf{(Required)} Register a GitHub account. Upload your repository of assignments onto GitHub. Keep pushing your commits to GitHub.
\end{thmquestion}

\begin{thmquestion}
\textbf{(Required for those responsible for topics)} Finish the following:
\begin{partlist}
\item Fork the seminar repository.
\item Clone the forked seminar repostitory.
\item Make commits to the forked seminar repository to add your outlines and materials.
\item Make pull request to the original repository.
\item Set an issue on the original seminar repository to collect suggestions.
\end{partlist}
\end{thmquestion}

\subsection{Remarks}

\textbf{The name of repository have been changed into \texttt{ToolsSeminar}, please rename your forked repository!!!}

\textbf{Once you are familiar enough with GitHub, contact me for collaborator access of the seminar repository.}

Any modifications of outline and materials are welcomed. Contact the author and push if you want to extend the topic of perform modifications.

The file \verb".gitignore.template" is a template for \verb".gitignore" file. You may use \verb"cp .gitignore.template .gitignore" to create your own .gitignore file. Adding new entries to the template is welcomed.

Further information about \verb".gitignore" can be found in \href{https://www.liaoxuefeng.com/wiki/0013739516305929606dd18361248578c67b8067c8c017b000/0013758404317281e54b6f5375640abbb11e67be4cd49e0000}{Xuefeng Liao's Git Tutorial} and \href{https://git-scm.com/docs/gitignore}{Git's reference}. GitHub provides a \href{https://github.com/github/gitignore}{repository} for several \verb".gitignore" templates.

	\end{document}
