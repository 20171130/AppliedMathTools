% !TeX encoding = UTF-8 Unicode
% !TeX program = LuaLaTeX
% !TeX spellcheck = LaTeX

% Author : pppppass
% Description : Matetials: Configuration -- Seminar on Selected Tools Week 0

\documentclass[english]{pkupaper}

\usepackage[paper]{def}

\usepackage{lipsum}

\newcommand{\cuniversity}{}
\newcommand{\cthesisname}{Matetials: Configuration -- Seminar on Selected Tools Week 0}
\newcommand{\titlemark}{Matetials: Configuration}

\title{\titlemark}
\author{pppppass}
\date{\today}

	\begin{document}

	\maketitle

This material provides some suggestions on system and environment-related issues and everything is optional. The information is updated on December 3, 2017.

\section{Environments}

\subsection{Installation and configuration}

It is suggested to use a UNIX-like environment when programming even if you have a Windows system. Reasons includes:
\begin{partlist}
\item There are differences between UNIX-like environments and Windows, including line feed (\verb"\n" in Linux but \verb"\n\r" in Windows), path separator (\verb"/" in Linux and \verb"\" in Windows) and so on.
\item Some packages does not work well in Windows.
\item Tool chains are broken up in Windows. That is, Git, TeX Live, Python, PuTTY all provide different environments in Windows, instead of a unified tool.
\item Getting accustomed to shells and terminals benefits because you may login to remote servers at some point.
\end{partlist}

There are several known approach to such environments for Windows users, including:
\begin{partlist}
\item Buy an Apple computer with Mac OS. (Expensive)
\item Switch to a Linux system. (Time-expensive)
\item Set up a dual system. (Probably unstable)
\item Install a virtual machine, and use the graphical interface. (Slow)
\item Use a terminal emulator in Windows. (Not fully functional)
\item Install a virtual machine and use a SSH client  to remote login into the virtual machine. (No programming support in windows)
\item Install a virtual machine and use a terminal emulator in Windows to remote login into the virtual machine. (Troublesome)
\item Use Ubuntu directly in Windows 10 from Microsoft Store. (Not popular)
\end{partlist}

For a virtual machine, I personally recommend \href{https://www.virtualbox.org/}{VirtualBox}, which is a free virtual machine program by Oracle. VMWare is another famours virtual machine software, but there are license issues. Additionally, VirtualBox provides better support for Linux systems, while VMWare is better for Windows systems.

For a Linux distributions, I personally recommend \href{https://www.ubuntu.com/desktop}{Ubuntu for desktops}. Ubuntu 16.04.3 LTS is more stable but somehow backward, while Ubuntu 17.10 is contrary.

For a terminal emulator in Windows, I personally recommend \href{http://www.msys2.org/}{MSYS2}, which provides some Linux utilities in Windows version.

For Linux and Mac OS users, nothing is needed but just running a terminal.

SSH and Vim are always included in Unix-like environments. If not, you may install it through package manager.

\subsection{Resources}

Downloads of VirtualBox can be found in \href{https://www.virtualbox.org/wiki/Downloads}{here}. Documentation can be found \href{https://www.virtualbox.org/wiki/Documentation}{here}, where a \verb".pdf" User Manual is provided.

Downloads of Ubuntu can be found in \href{https://www.ubuntu.com/download/desktop}{here}. There are easy-to-understand instructions during installation.

Tutorials of instaling Ubuntu on a VirtualBox virtual machine can be found very easily by search engines.

Downloads of MSYS2 can be found in \href{http://www.msys2.org/}{here} and a small installation guide is also included. Futher tutorials can also be found on Internet.

Ubuntu officially provides a tutorial \href{https://tutorials.ubuntu.com/tutorial/tutorial-ubuntu-on-windows}{Install Ubuntu on Windows 10}.

\href{https://linux.cn/article-6160-1.html}{This article} is a Chinese introduction to Linux command lines.

NIH provides a course \href{https://hpc.nih.gov/training/handouts/Linux_NIH_2017.pdf}{\emph{Introduction to Linux}} to tell basic GNU and Linux concepts. The first 32 pages is adequate for a start in Linux.

To perform remote login, you may needs to generate a SSH key. This is described in Xuefeng Liao's Git tutorial in the section \href{https://www.liaoxuefeng.com/wiki/0013739516305929606dd18361248578c67b8067c8c017b000/001374385852170d9c7adf13c30429b9660d0eb689dd43a000}{\emph{Remote repositories}} and in Git official tutorial in section 4.3 \href{https://git-scm.com/book/en/v2/Git-on-the-Server-Generating-Your-SSH-Public-Key}{\emph{Git on the Server -- Generating Your SSH Public Key}}.

If you want to permit remote logins into your system, a SSH service is also required, which is described \href{http://www.linuxidc.com/Linux/2010-02/24349.htm}{here}.

\href{https://stackoverflow.com/questions/11828270/how-to-exit-the-vim-editor}{This post} in Stack Overflow describes how to exit the Vim editor.

\href{http://www.jianshu.com/p/bcbe916f97e1}{This post} and \href{https://blog.interlinked.org/tutorials/vim_tutorial.html}{this website} give a complete introduction to Vim.

\section{Anaconda}

\subsection{Installation and configuration}

It is strongly recommended to install and use Anaconda in a UNIX-like environmnet.

Anaconda can be directly downloaded from \href{https://www.anaconda.com/download/}{official website}.

During the installation, I personally prefer not to add the path of Anaconda into \verb"~/.bashrc" to avoid overriding Python utilities in the original system. However, if no path is added, you have to explicitly use \verb"source some/path/to/anaconda/bin/activate" to activate the Anaconda.

\subsection{Resources}

\href{https://conda.io/docs/user-guide/index.html}{Anaconda's User Guide} gives a great introduction to the usage of Anaconda. The section \href{https://conda.io/docs/user-guide/getting-started.html}{\emph{Getting started}} shows basic usage of the command \verb"conda", and the section \href{https://conda.io/docs/user-guide/cheatsheet.html}{\emph{Cheet sheet}} gives a list of \verb"conda" commands.

\href{http://www.jianshu.com/p/169403f7e40c}{This post} is a Chinese guide to Anaconda.

You may search \href{https://anaconda.org/}{Anaconda Cloud} to search for non-standard package, e.g. MOSEK, Gurobi, CVXPY and many more.

\section{Jupyter Notebook}

\subsection{Installation and configuration}

It is strongly recommended to install and use Jupyter Notebook in Anaconda.

In Anaconda, you can install jupyter notebook by \verb"conda install jupyter".

You may consult \href{https://jupyter.readthedocs.io/en/latest/running.html}{\emph{Running the Notebook}} to run a Jupyter Notebook. For public access, you may generate a config file as shown in \href{https://jupyter-notebook.readthedocs.io/en/stable/config_overview.html}{\emph{Configuration Overview}}, and then set password as \href{https://jupyter-notebook.readthedocs.io/en/stable/public_server.html}{\emph{Running a Notebook Server}}. Using SSL for encryption takes some time and is not much necessary is you run the Notebook in local network.

\subsection{Resources}

\href{https://jupyter.readthedocs.io/en/latest/content-quickstart.html}{\emph{Jupyter Notebook Quickstart}} is a official tutorial of Jupyter Notebook. Only the \emph{Running the Notebook} section is in necessity.

\href{http://blog.csdn.net/bitboy_star/article/details/51427306}{This post} is a brief tutorial on remote access of Jupyter Notebook in Chinese.

\href{http://blog.csdn.net/mzpmzk/article/details/72310105}{This post} is a complete introduction to IPython and Jupyter Notebook. \href{https://www.packtpub.com/books/content/getting-started-jupyter-notebook-part-1}{This post} and \href{https://www.packtpub.com/books/content/getting-started-jupyter-notebook-part-2}{this post} introduc Jupyter Notebook, while \href{http://blog.csdn.net/red_stone1/article/details/72858962}{this post} and \href{http://blog.csdn.net/red_stone1/article/details/72863749}{this post} are Chinese translations.

\href{https://ipython.readthedocs.io/en/stable/interactive/tutorial.html#magics-explained}{\emph{Introduction IPython}} is an introduction to IPython, an interactive Python prompt which Jupyter Notebook is based on. Magic commands are mentioned here, while two of the most important magic commands in Jupyter notebook are \verb"matplotlib notebook" (show interactive plots) and \verb"matplotlib inline" (show inline static plots).

Jupyter Notebook is a versatile tool for interactive Python programming, and further information can be found in \href{https://jupyter-notebook.readthedocs.io/en/stable/index.html}{Jupyter Notebook document} and \href{https://ipython.readthedocs.io/en/stable/index.html}{IPython document}.

	\end{document}
