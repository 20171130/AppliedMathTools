\subsection{serialization: pickle and json}
You can get an enlightening introduction about how to use these two modules in \href{https://www.liaoxuefeng.com/wiki/0014316089557264a6b348958f449949df42a6d3a2e542c000/00143192607210600a668b5112e4a979dd20e4661cc9c97000}{this page} .
\subsubsection{pickle}
\texttt{pickle} implement binary protocols for serializing and de-serializing a Python object structure. \textit{“Pickling”} is the process whereby a Python object hierarchy is converted into a byte stream, and \textit{“unpickling”} is the inverse operation, whereby a byte stream (from a binary file or bytes-like object) is converted back into an object hierarchy. 

The pickle module provides the following functions to make the pickling process more convenient:
\begin{itemize}
\item \textbf{dump}(obj, file, protocol=None, *, fix\verb?_?imports=True) writes a pickled representation of \textit{obj} to the open file object file.
\item \textbf{dumps}(obj, protocol=None, *, fix\verb?_?imports=True) returns the pickled representation of the object as a bytes object, instead of writing it to a file.
\item \textbf{load}(file, *, fix\verb?_?imports=True, encoding="ASCII", errors="strict") reads a pickled object representation from the open file object file and return the reconstituted object hierarchy specified therein.
\item \textbf{loads}(bytes\verb?_?object, *, fix\verb?_?imports=True, encoding="ASCII", errors="strict") reads a pickled object hierarchy from a bytes object and return the reconstituted object hierarchy specified therein.
\end{itemize} 
The pickle module exports two classes, Pickler and Unpickler:
\begin{itemize}
\item class \texttt{pickle}.\textbf{Pickler}(file, protocol=None, *, fix\verb?_?imports=True)

This takes a binary file for writing a pickle data stream.

\textbf{dump}$(obj)$ writes a pickled representation of obj to the open file object given in the constructor.

\item class \texttt{pickle}.\textbf{Unpickler}(file, *, fix\verb?_?imports=True, encoding="ASCII", errors="strict")

This takes a binary file for reading a pickle data stream.

\textbf{load}() reads a pickled object representation from the open file object given in the constructor, and returns the reconstituted object hierarchy specified therein. Bytes past the pickled object’s representation are ignored.
\end{itemize}
The types that can be pickled can be found in \href{https://docs.python.org/3.6/library/pickle.html#what-can-be-pickled-and-unpickled}{this page}.

\subsubsection{json}
\texttt{json}  is a lightweight data interchange format inspired by JavaScript object literal syntax.

basic usage:
\begin{itemize}
\item \textbf{dump}(obj, fp, *, skipkeys=False, ensure\_ascii=True, check\_circular=True, allow\_nan=True, cls=None, indent=None, separators=None, default=None, sort\_keys=False, **kw) serializes obj as a JSON formatted stream to fp (a .write()-supporting file-like object) using this \href{https://docs.python.org/3.6/library/json.html#py-to-json-table}{conversion table}.

All optional parameters are now \textit{keyword-only}.
\item \textbf{dumps}(obj, *, skipkeys=False, ensure\_ascii=True, check\_circular=True, allow\_nan=True, cls=None, indent=None, separators=None, default=None, sort\_keys=False, **kw) serializes obj to a JSON formatted str.

The arguments have the same meaning as in \textit{dump()}.
\item \textbf{load}(fp, *, cls=None, object\_hook=None, parse\_float=None, parse\_int=None, parse\_constant=None, object\_pairs\_hook=None, **kw) deserializes fp (a .read()-supporting file-like object containing a JSON document) to a Python object.

All optional parameters are now \textit{keyword-only}.
\item \textbf{loads}(s, *, encoding=None, cls=None, object\_hook=None, parse\_float=None, parse\_int=None, parse\_constant=None, object\_pairs\_hook=None, **kw) deserializes s (a \textit{str, bytes} or \textit{bytearray} instance containing a JSON document) to a Python object.

The other arguments have the same meaning as in \text{load()}, except \textit{encoding} which is ignored and deprecated.
\end{itemize}
Besides, we can serialize a class object through utilizing other selective arguments in method \textit{dumps()}. Detailed instruction can be found in \href{https://docs.python.org/3/library/json.html#json.dumps}{here}.